\chapter{Specifikacija programske potpore}
		
	\section{Funkcionalni zahtjevi}
			
			\textit{ \textbf{Dionici} koji imaju interesa u ovom sustavu su svi \textbf{doktori, treneri i budući korisnici sustava},
			te dodatno \textbf{naručitelj projekta} (u ovom slučaju asistent tj. fakultet). Nositelji odgovornosti su \textbf{budući administratori sustava te razvojni tim}.}\\
				
			\textit{\textbf{Aktori} koji izravno koriste sustav su svi \textbf{doktori, treneri, klijenti i administratori}, dok sa sustavom neizravno komuniciraju \textbf{neregistrirani korisnici te baza podataka}.}\\
			
			
			\noindent \textbf{Dionici:}
			
			\begin{packed_enum}
				
				\item 1. Klijenti
				\item 2. Doktori			
				\item 3. Treneri
				\item 4. Razvojni tim (grupa Flow)
				\item 5. Naručitelj (asistent)
				\item 6. Administratori aplikacije
				
			\end{packed_enum}
			
			\noindent \textbf{Aktori i njihovi funkcionalni zahtjevi:}
			
			
			\begin{packed_enum}
				\item  \underbar{Administrator (inicijator) može:}
				
				\begin{packed_enum}
					
					\item potvrditi ili odbiti željenu ulogu neregistriranom korisniku

					\item dodavati proizvode i vježbe u kategorije
					
					\item uređivati i brisati
					\begin{packed_enum}
						
						\item kategorije
						\item proizvode
						\item vježbe
						
					\end{packed_enum}
				\end{packed_enum}
			
				\item  \underbar{Doktor (inicijator) može:}
				
				\begin{packed_enum}
					
					\item odgovarati na recenzije svojih klijenata
					\item prekinuti suradnju s klijentom
					\item definirati dijetu
					\begin{packed_enum}
						
						\item unositi sve potrebne parametre dijete
						\item dodati opis dijete
						
					\end{packed_enum}
				
					\item pregledati statistike klijenta po danu
					\item dodavati proizvode i vježbe u kategorije	
					
				\end{packed_enum}
			
				\item  \underbar{Trener (inicijator) može:}
				
				\begin{packed_enum}
					
					\item odgovarati na recenzije svojih klijenata
					\item prekinuti suradnju s klijentom
					\item definirati dnevne treninge
					\item pregledati statistike klijenta po danu
					\item dodavati proizvode i vježbe u kategorije
					
				\end{packed_enum}
			
				\item  \underbar{Klijent (inicijator) može:}
				
				\begin{packed_enum}
					
					\item tražiti i pregledavati profile svih dostupnih doktora i trenera
					\item ostaviti recenziju svom doktoru i treneru
					\item prekinuti suradnju sa svojim doktorom ili trenerom
					\item pregledati vlastite statistike po danu
					\item skenirati bar kod proizvoda koje planira konzumirati
					
				\end{packed_enum}
			
				\item  \underbar{Neregistrirani korisnik (inicijator) može:}
				
				\begin{packed_enum}
					
					\item poslati zahtjev za registraciju sa željenom ulogom
					
					\begin{packed_enum}
						
						\item klijent
						\item trener
						\item doktor	
						
					\end{packed_enum}
					
				\end{packed_enum}
			
				\item  \underbar{Baza podataka (sudionik) može:}
				
				\begin{packed_enum}
					
					\item pohranjuje podatke o svim: 
					
					\begin{packed_enum}
						
						\item registriranim osobama
						\item recenzijama
						\item dijetama i proizvodima
						\item treninzima i vježbama
					
					\end{packed_enum}	
				
				\end{packed_enum}
			
			\end{packed_enum}
			
			\eject 
			
			
				
			\subsection{Obrasci uporabe}
				
				\textbf{\textit{dio 1. revizije}}
				
				\subsubsection{Opis obrazaca uporabe}
					\textit{Funkcionalne zahtjeve razraditi u obliku obrazaca uporabe. Svaki obrazac je potrebno razraditi prema donjem predlošku. Ukoliko u nekom koraku može doći do odstupanja, potrebno je to odstupanje opisati i po mogućnosti ponuditi rješenje kojim bi se tijek obrasca vratio na osnovni tijek.}\\
					

					\noindent \underbar{\textbf{UC1 - Registracija}}
					\begin{packed_item}
	
						\item \textbf{Glavni sudionik: }Neregistrirani korisnik
						\item  \textbf{Cilj:} Stvoriti korinički račun za pristup sustavu
						\item  \textbf{Sudionici:} Baza podataka
						\item  \textbf{Preduvjet:} -
						\item  \textbf{Opis osnovnog tijeka:}
						
						\item[] \begin{packed_enum}
	
							\item Korisnik odabire opciju za registraciju
							\item Korisnik unosi potrebne podatke
							\item Korisnik čeka na potvrdu administratora
						\end{packed_enum}
						
						\item  \textbf{Opis mogućih odstupanja:}
						
						\item[] \begin{packed_item}
	
							\item[2.a] $<$opis mogućeg scenarija odstupanja u koraku 2$>$
							\item[] \begin{packed_enum}
								
								\item Sustav obavještava korisnika o neuspjelom unosu i vraća ga na stranicu za registraciju
								\item Korisnik mijenja potrebne podatke te završava unos ili odustaje od registracije
								
							\end{packed_enum}
							
						\end{packed_item}
					\end{packed_item}
				
					\noindent \underbar{\textbf{UC$<$broj obrasca$>$ -$<$ime obrasca$>$}}
					\begin{packed_item}
	
						\item \textbf{Glavni sudionik: }$<$sudionik$>$
						\item  \textbf{Cilj:} $<$cilj$>$
						\item  \textbf{Sudionici:} $<$sudionici$>$
						\item  \textbf{Preduvjet:} $<$preduvjet$>$
						\item  \textbf{Opis osnovnog tijeka:}
						
						\item[] \begin{packed_enum}
	
							\item $<$opis korak jedan$>$
							\item $<$opis korak dva$>$
							\item $<$opis korak tri$>$
							\item $<$opis korak četiri$>$
							\item $<$opis korak pet$>$
						\end{packed_enum}
						
						\item  \textbf{Opis mogućih odstupanja:}
						
						\item[] \begin{packed_item}
	
							\item[2.a] $<$opis mogućeg scenarija odstupanja u koraku 2$>$
							\item[] \begin{packed_enum}
								
								\item $<$opis rješenja mogućeg scenarija korak 1$>$
								\item $<$opis rješenja mogućeg scenarija korak 2$>$
								
							\end{packed_enum}
							\item[2.b] $<$opis mogućeg scenarija odstupanja u koraku 2$>$
							\item[3.a] $<$opis mogućeg scenarija odstupanja  u koraku 3$>$
							
						\end{packed_item}
					\end{packed_item}
				
					
				\subsubsection{Dijagrami obrazaca uporabe}
					
					\textit{Prikazati odnos aktora i obrazaca uporabe odgovarajućim UML dijagramom. Nije nužno nacrtati sve na jednom dijagramu. Modelirati po razinama apstrakcije i skupovima srodnih funkcionalnosti.}
				\eject		
				
			\subsection{Sekvencijski dijagrami}
				
				\textbf{\textit{dio 1. revizije}}\\
				
				\textit{Nacrtati sekvencijske dijagrame koji modeliraju najvažnije dijelove sustava (max. 4 dijagrama). Ukoliko postoji nedoumica oko odabira, razjasniti s asistentom. Uz svaki dijagram napisati detaljni opis dijagrama.}
				\eject
	
		\section{Ostali zahtjevi}
		
			\textbf{\textit{dio 1. revizije}}\\
		 
			 \textit{Nefunkcionalni zahtjevi i zahtjevi domene primjene dopunjuju funkcionalne zahtjeve. Oni opisuju \textbf{kako se sustav treba ponašati} i koja \textbf{ograničenja} treba poštivati (performanse, korisničko iskustvo, pouzdanost, standardi kvalitete, sigurnost...). Primjeri takvih zahtjeva u Vašem projektu mogu biti: podržani jezici korisničkog sučelja, vrijeme odziva, najveći mogući podržani broj korisnika, podržane web/mobilne platforme, razina zaštite (protokoli komunikacije, kriptiranje...)... Svaki takav zahtjev potrebno je navesti u jednoj ili dvije rečenice.
			 Nefunkcionalni zahtjevi
			 1) Pristup aplikaciji mora biti omogućen iz javne mreže pomoću HTTPS-a.
			 2) Veza s bazom podataka mora biti sigurna i kvalitetna. Također veza mora biti brza.
			 3) Nadogradnja sustava mora biti moguća.
			 4) Sustav mora biti jednostavan za korištenje.
			 5) Kratke upute o korištenju moraju biti omogućene.
			 6) Obavijestiti korisnika o neispravnom korištenju sučelja.
			 7) Neispravno korištenje sučelja ne smije narušiti rad aplikacije.
			 8) Sustav mora biti implementiran kao web-aplikacija pomoću objektno orijentiranih jezika.
			 9) Sustav mora podržavati rad više korisnika u istome trenutku.
			 10) Aplikacija mora podržavati hrvatsku abecedu.}\\
			 
			 
			 
	
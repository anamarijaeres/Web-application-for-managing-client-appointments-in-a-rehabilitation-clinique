\chapter{Arhitektura i dizajn sustava}
		
		\textbf{\textit{dio 1. revizije}}\\

		\textit{Arhitektura se sastoji od 3 podsustava:}
	\begin{itemize}
		\item 	\textit{Web poslužitelj}
		\item 	\textit{Web aplikacija}
		\item 	\textit{Baza podataka}		
	\end{itemize}
		\textit{
			Web poslužitelj- Primarna zadaća web poslužitelja je komunikacija klijenta s aplikacijom, pomoću HTTP protokola. Web poslužitelj prosljeđuje korisnikov 
			zahtjev, preko Web preglednika, web aplikaciji. 
			Web aplikacija- obrađuje zahtjev te pristupa bazi podataka, vraćajući odgovor korisniku u web pregledniku.
			Baza podataka- implementacija pomoću sql-a.
			Za izradu frontend dijela aplikacije ćemo koristiti Java Script, a za izradu backend dijela aplikacije ćemo koristiti Java Spring Boot. 
			DOVRŠI MVC koncept arhitekture}\\
			
		\section{Baza podataka}
			
			\textbf{\textit{dio 1. revizije}}\\
			Za potrebe našeg sustava koristit ćemo relacijsku bazu podataka koja svojom strukturom olakšava modeliranje stvarnog svijeta. Gradivna jedinka baze je relacija, odnosno tablica koja je definirana svojim imenom i skupom atributa. Zadaća baze podataka je brza i jednostavna pohrana, izmjena i dohvat podataka za daljnju obradu.
			Baza podataka ove aplikacije sastoji se od sljedećih entiteta:
			\begin{itemize}
				\item {Korisnik}
				\item {Doktori-Treneri}
				\item {Recenzije}
				\item {Klijent-Doktor}
				\item {Klijent-Trener}
				\item {KategorijeProizvoda}
				\item {Proizvod}
				\item {NutritivneVrijednosti}
				\item {Vježba}
				\item {Intenzitet}
				\item {BrPotrošenihKalorija}
				\item {Trening}
				\item {NizVježbi}
				\item {Dijeta}
				\item {OgraničenjaNaProizvode}
				\item {OgraničenjaNaKategorije}
				\item {DnevniLimit}
				\item {UneseneNutrVrijednosti}
				\item {KonzumiraniProizvodi}
			\end{itemize}
			
			
		
			\subsection{Opis tablica}
			

				\textbf{Korisnik} Ovaj entitet sadržava sve važne informacije o korisniku aplikacije. Svaki korisnik ima korisničko ime, lozinku, ime, prezime i naziv uloge. Entitet Korisnik je preko atributa KorisnickoIme u \textit{One-to-One} vezi s DoktoriTreneri, \textit{One-to-One} vezi s KlijentDoktor, \textit{One-to-One} vezi s KlijentTrener, \textit{One-to-Many} vezi s entitetom Trening, \textit{One-to-Many} vezi s UneseneNutrVrijednosti, \textit{One-to-Many} vezi s KonzumiraniProizvodi, \textit{One-to-Many} vezi s entitetom Dijeta te \textit{One-to-Many} vezi s entitetom Recenzija.
				
				
				
				
				\begin{longtabu} to \textwidth {|X[8, l]|X[6, l]|X[20, l]|}
					
					\hline \multicolumn{3}{|c|}{\textbf{Korisnik}}	 \\[3pt] \hline
					\endfirsthead
					
					\hline \multicolumn{3}{|c|}{\textbf{Korisnik}}	 \\[3pt] \hline
					\endhead
					
					\hline 
					\endlastfoot
					
					\colorbox{LightGreen}{KorisnickoIme} & VARCHAR	&  jedinstveni identifikator korisnika\\ \hline
					Ime	& VARCHAR &  ime korisnika 	\\ \hline 
					Prezime & VARCHAR &  prezime korisnika \\ \hline 
					Lozinka & VARCHAR	& lozinka korisnika 		\\ \hline 
					Uloga	& VARCHAR & Uloga korisnika: doktor, trener ili klijent.  	\\ \hline 
					
					
				\end{longtabu}
				
				\textbf{DoktoriTreneri} Da bi neregistrirani korisnik dobio prava doktora i trenera, administrator ga mora potvrditi. Potrebno je dodatno priložiti sliku, email i maksimalni broj korisnika koje želi nadgledati. Navedeni atributi se spremaju u entitet DoktoriTreneri. Entitet DoktoriTreneri je preko atributa KorisnickoIme u \textit{One-to-One} vezi s entitetom Korisnik, \textit{One-to-Many} vezi s KlijentDoktor, \textit{One-to-Many} vezi s KlijentTrener te \textit{One-to-Many} vezi s entitetom Recenzije.
				
				
				
				\begin{longtabu} to \textwidth {|X[8, l]|X[6, l]|X[20, l]|}
					
					\hline \multicolumn{3}{|c|}{\textbf{DoktoriTreneri}}	 \\[3pt] \hline
					\endfirsthead
					
					\hline \multicolumn{3}{|c|}{\textbf{DoktoriTreneri}}	 \\[3pt] \hline
					\endhead
					
					\hline 
					\endlastfoot
					
					\colorbox{LightGreen}{korisnickoIme} & VARCHAR	&  jedinstveni identifikator korisnika\\ \hline
					Mail   & VARCHAR &  e-mail adresa doktora/trenera 	\\ \hline 
					Slika  & BYTEA &  fotografija doktora/trenera \\ \hline 
					BrojKlijenata& INT	& trenutni broj prijavljenih klijenata 		\\ \hline 
					MaxBrKlijenata	& INT & maksimalan broj prijavljenih klijenata  	\\ \hline 
					
					
				\end{longtabu}
				
				\textbf{Recenzije} Klijent svom doktoru i treneru na profilu može ostaviti recenziju s ocjenom i komentarom, a doktor i trener mogu odgovoriti na vlastitu recenziju. Entitet Recenzije sadrži atribute: korisnickoIme, korImeKlijenta, ocjena, komentar, odgovor i ID komentara. Ovaj entitet je preko atributa KorImeKlijenta u \textit{Many-to-One} vezi s Korisnik, te preko atributa KorisnickoIme u \textit{Many-to-One} vezi s entitetom DoktoriTreneri.
				
				\begin{longtabu} to \textwidth {|X[9, l]|X[6, l]|X[20, l]|}
					
					\hline \multicolumn{3}{|c|}{\textbf{Recenzije}}	 \\[3pt] \hline
					\endfirsthead
					
					\hline \multicolumn{3}{|c|}{\textbf{Recenzije}}	 \\[3pt] \hline
					\endhead
					
					\hline 
					\endlastfoot
					\colorbox{LightGreen}{ID} & INT & jedinstveni identifikator recenzije \\ \hline
					\colorbox{LightBlue}{KorImeKlijenta} & VARCHAR	&  jedinstveni identifikator klijenta \\ \hline
					\colorbox{LightBlue}{KorisnickoIme} & VARCHAR & jedinstveni identifikator doktora/trenera 	\\ \hline 
					Ocjena  & INT &  ocjena od 1 do 5 \\ \hline 
					Komentar & VARCHAR	& recenzija doktora/trenera 		\\ \hline 
					Odgovor	& VARCHAR & odgovor doktora/trenera na recenziju  	\\ \hline
					
					
					
				\end{longtabu}
				
				\textbf{KlijentDoktor} Entitet sadrži informacije o trenutnoj suradnji klijenta i nekog doktora. Atributi su: korImeKlijenta i korImeDoktora. Ovaj entitet je preko atributa KorImeKlijenta u \textit{One-to-One} vezi s Korisnik te preko atributa korImeDoktora u \textit{Many-to-One} vezi s entitetom DoktoriTreneri.
				
				\begin{longtabu} to \textwidth {|X[9, l]|X[6, l]|X[20, l]|}
					
					\hline \multicolumn{3}{|c|}{\textbf{KlijentDoktor}}	 \\[3pt] \hline
					\endfirsthead
					
					\hline \multicolumn{3}{|c|}{\textbf{KlijentDoktor}}	 \\[3pt] \hline
					\endhead
					
					\hline 
					\endlastfoot
					
					\colorbox{LightGreen}{KorImeKlijenta} & VARCHAR	& jedinstveni identifikator klijenta \\ \hline
					\colorbox{LightBlue}{KorImeDoktora} & VARCHAR & jedinstveni identifikator doktora\\ \hline 
					
				\end{longtabu}
				
				\textbf{KlijentTrener} Entitet sadrži informacije o trenutnoj suradnji klijenta i nekog trenera. Atributi su: korImeKlijenta i korImeTrenera. Ovaj entitet je preko atributa KorImeKlijenta u \textit{One-to-One} vezi s Korisnik te preko atributa korImeTrenera u \textit{Many-to-One} vezi s entitetom DoktoriTreneri.
				
				\begin{longtabu} to \textwidth {|X[9, l]|X[6, l]|X[20, l]|}
					
					\hline \multicolumn{3}{|c|}{\textbf{KlijentTrener}}	 \\[3pt] \hline
					\endfirsthead
					
					\hline \multicolumn{3}{|c|}{\textbf{KlijentTrener}}	 \\[3pt] \hline
					\endhead
					
					\hline 
					\endlastfoot
					
					\colorbox{LightGreen}{KorImeKlijenta} & VARCHAR	&  jedinstveni identifikator klijenta \\ \hline
					\colorbox{LightBlue}{KorImeTrenera} & VARCHAR & jedinstveni identifikator trenera \\ \hline 
					
				\end{longtabu}
				
				\textbf{KategorijeProizvoda} Kategorija proizvoda sadrži više različitih proizvoda sličnih karakteristika, kao što su na primjer tjestenina, proizvodi od mlijeka, meso itd. Entitet KategorijeProizvoda sadrži atribute: ID kategorije i ime kategorije. Ovaj entitet je preko atributa ID u \textit{Many-to-Many} vezi s entitetom Proizvod te \textit{Many-to-Many} vezi s entitetom OgranicenjaNaKategorije.
				
				\begin{longtabu} to \textwidth {|X[7, l]|X[6, l]|X[20, l]|}
					
					\hline \multicolumn{3}{|c|}{\textbf{KategorijeProizvoda}}	 \\[3pt] \hline
					\endfirsthead
					
					\hline \multicolumn{3}{|c|}{\textbf{KategorijeProizvoda}}	 \\[3pt] \hline
					\endhead
					
					\hline 
					\endlastfoot
					
					\colorbox{LightGreen}{ID} & INT	&  jedinstveni identifikator kategorije \\ \hline
					Ime & VARCHAR & ime kategorije 	\\ \hline 
					
				\end{longtabu}
				
				\textbf{Proizvod} Svaki proizvod sadrži sliku, masu i prisutne alergene. Svaki proizvod pripada nekoj kategoriji proizvoda. Entitet Proizvod sadrži atribute: ID proizvoda, masa, slika, barkod, prisutni alergeni, ime te ID kategorije. Ovaj entitet je preko ID atributa u \textit{Many-to-Many} vezi s KategorijeProizvoda, \textit{Many-to-Many} vezi s OgranicenjaNaProizvode, \textit{One-to-One} vezi s NutritivneVrijednosti te \textit{Many-to-Many} vezi s KonzumiraniProizvodi. 
				
				\begin{longtabu} to \textwidth {|X[7, l]|X[6, l]|X[20, l]|}
					
					\hline \multicolumn{3}{|c|}{\textbf{Proizvod}}	 \\[3pt] \hline
					\endfirsthead
					
					\hline \multicolumn{3}{|c|}{\textbf{Proizvod}}	 \\[3pt] \hline
					\endhead
					
					\hline 
					\endlastfoot
					
					\colorbox{LightGreen}{ID} & INT	&  jedinstveni identifikator proizvoda \\ \hline
					Ime & VARCHAR & ime proizvoda 	\\ \hline 
					Masa & DOUBLE & masa proizvoda u gramima\\ \hline
					Slika & BYTEA & slika proizvoda\\ \hline
					Barkod & BYTEA & barkod proizvoda\\ \hline
					Alergeni & VARCHAR & popis prisutnih alergena\\ \hline
					\colorbox{LightBlue}{IDkategorije} & INT & ID kategorije proizvoda\\ \hline 
					
					
				\end{longtabu}
				
				
				\textbf{NutritivneVrijednosti} Informacije o nutritivnim vrijednostima proizvoda su definirane na 100g, a to su energija, masnoće, zasićene masne kiseline, ugljikohidrati, šećeri, bjelančevine i sol. Entitet NutritivneVrijednosti sadrži istoimene atribute te dodatno ID proizvoda za koji su vrijednosti definirane. Ovaj entitet je preko atributa IDproizvoda u \textit{One-to-One} vezi s entitetom Proizvod.  
				
				\begin{longtabu} to \textwidth {|X[10, l]|X[6, l]|X[20, l]|}
					
					\hline \multicolumn{3}{|c|}{\textbf{NutritivneVrijednosti}}	 \\[3pt] \hline
					\endfirsthead
					
					\hline \multicolumn{3}{|c|}{\textbf{NutritivneVrijednosti}}	 \\[3pt] \hline
					\endhead
					
					\hline 
					\endlastfoot
					
					\colorbox{LightGreen}{IDproizvoda} & INT & jedinstveni identifikator proizvoda \\ \hline
					Energija & DOUBLE & energija u kJ 	\\ \hline 
					Masnoća & DOUBLE & masnoća u sastavu proizvoda\\ \hline
					ZasMasneKiseline & DOUBLE & zasićene masne kiseline u sastavu proizvoda\\ \hline
					Ugljikohidrati & DOUBLE & ugljikohidrati u sastavu proizvoda\\ \hline
					Šećeri & DOUBLE & šećeri u sastavu proizvoda\\ \hline
					Bjelančevine & DOUBLE & bjelančevine u sastavu proizvoda\\ \hline
					Sol & DOUBLE & soli u sastavu proizvoda\\ \hline	
					
					
				\end{longtabu}
				
				\textbf{Vježba} Vježba je definirana sa slikom, opisom i informacijama o broju potrošenih kalorija u sat vremena u ovisnosti o 3 razine intenziteta vježbanja (lagano, normalno, teško). Entitet Vježba dodatno sadrži atribut ID, tj. jedinstveni identifikator vježbe. Ovaj entitet je preko atributa ID u \textit{One-to-Many} vezi s entitetom NizVjezbi te \textit{One-to-Many} vezi s entitetom BrPotrosenihKalorija.
				
				\begin{longtabu} to \textwidth {|X[7, l]|X[6, l]|X[20, l]|}
					
					\hline \multicolumn{3}{|c|}{\textbf{Vježba}}	 \\[3pt] \hline
					\endfirsthead
					
					\hline \multicolumn{3}{|c|}{\textbf{Vježba}}	 \\[3pt] \hline
					\endhead
					
					\hline 
					\endlastfoot
					
					\colorbox{LightGreen}{ID} & INT	&  jedinstveni identifikator vježbe \\ \hline
					Slika & BYTEA & slika vježbe 	\\ \hline 
					Opis & VARCHAR & opis vježbe\\ \hline
					
					
				\end{longtabu}
				\textbf{Intenzitet} Entitet Intenzitet se sastoji od atributa: razina intenziteta i šifra intenziteta. Intenzitet može biti: lagan, normalan i težak. Intenzitet je preko atributa Razina u \textit{Many-to-Many} vezi s BrPotrosenihKalorija te \textit{One-to-Many} vezi s entitetom NizVjezbi.
				
				\begin{longtabu} to \textwidth {|X[7, l]|X[6, l]|X[20, l]|}
					
					\hline \multicolumn{3}{|c|}{\textbf{Intenzitet}}	 \\[3pt] \hline
					\endfirsthead
					
					\hline \multicolumn{3}{|c|}{\textbf{Intenzitet}}	 \\[3pt] \hline
					\endhead
					
					\hline 
					\endlastfoot
					
					\colorbox{LightGreen}{Šifra} & INTEGER & šifra intenziteta\\ \hline 
					Razina & VARCHAR	&  intenzitet: lagano, normalno ili teško \\ \hline
					
					
				\end{longtabu}
				
				\textbf{BrPotrošenihKalorija} Za svaku vježbu je definiran broj potrošenih kalorija u sat vremena u ovisnosti o 3 razine intenziteta vježbanja (lagano, normalno, teško). Ovaj entitet je preko atributa IDvjezbe u \textit{Many-to-One} vezi s entitetom Vjezba te preko atributa Intenzitet u \textit{Many-to-Many} vezi s entitetom Intenzitet. 
				
				\begin{longtabu} to \textwidth {|X[10, l]|X[6, l]|X[20, l]|}
					
					\hline \multicolumn{3}{|c|}{\textbf{BrPotrošenihKalorija}}	 \\[3pt] \hline
					\endfirsthead
					
					\hline \multicolumn{3}{|c|}{\textbf{BrPotrošenihKalorija}}	 \\[3pt] \hline
					\endhead
					
					\hline 
					\endlastfoot
					
					\colorbox{LightGreen}{IDvježbe} & INT	&  jedinstveni identifikator vježbe \\ \hline
					\colorbox{LightGreen}{Intenzitet} & VARCHAR & jedinstveni identifikator intenziteta\\ \hline 
					PotrošeneKalorije & DOUBLE & broj potrošenih kalorija u sat vremena
					
					
				\end{longtabu}
				
				
				\textbf{Trening} Trener zadaje trening klijentu svaki dan. Entitet Trening se sastoji od atributa: ID treninga, ID klijenta te datuma. Ovaj entitet je preko atributa IDklijenta u \textit{Many-to-One} vezi s entitetom Korisnik te preko atributa ID u \textit{One-to-Many} vezi s entitetom NizVjezbi.
				
				\begin{longtabu} to \textwidth {|X[7, l]|X[6, l]|X[20, l]|}
					
					\hline \multicolumn{3}{|c|}{\textbf{Trening}}	 \\[3pt] \hline
					\endfirsthead
					
					\hline \multicolumn{3}{|c|}{\textbf{Trening}}	 \\[3pt] \hline
					\endhead
					
					\hline 
					\endlastfoot
					
					\colorbox{LightGreen}{ID} & INT	&  jedinstveni identifikator treninga \\ \hline
					\colorbox{LightBlue}{IDklijenta} & VARCHAR & jedinstveni identifikator klijenta\\ \hline 
					Datum & DATETIME & datum treninga\\ \hline
					Odrađen & BOOLEAN & klijent mora potvrditi da je odradio trening\\ \hline
					
					
				\end{longtabu}
				
				\textbf{NizVježbi} Trening se sastoji od niza vježbi koji imaju određeno trajanje i intenzitet. Entitet NizVježbi se sastoji od atributa: ID treninga, ID vježbe, intenzitet vježbe, trajanje vježbe te redni broj vježbe. Ovaj entitet je preko atributa IDtreninga u \textit{Many-to-One} vezi s entitetom Trening, preko atributa IDvjezbe u \textit{Many-to-One} vezi s entitetom Vjezba te u \textit{Many-to-One} vezi s entitetom Intenzitet preko atributa Intenzitet.
				
				\begin{longtabu} to \textwidth {|X[7, l]|X[6, l]|X[20, l]|}
					
					\hline \multicolumn{3}{|c|}{\textbf{NizVježbi}}	 \\[3pt] \hline
					\endfirsthead
					
					\hline \multicolumn{3}{|c|}{\textbf{NizVježbi}}	 \\[3pt] \hline
					\endhead
					
					\hline 
					\endlastfoot
					
					\colorbox{LightGreen}{IDtreninga} & INT	&  jedinstveni identifikator treninga \\ \hline
					\colorbox{LightGreen}{RedniBroj} & INT & redni broj vježbe\\ \hline
					\colorbox{LightBlue}{IDvježbe} & INT & jedinstveni identifikator vježbe\\ \hline
					\colorbox{LightBlue}{Intenzitet} & VARCHAR & intenzitet vježbe (lagano, normalno ili teško)\\ \hline
					Trajanje & INTERVAL & trajanje vježbe\\ \hline
					
					
					
					
				\end{longtabu}
				
				\textbf{Dijeta} Entitet Dijeta se sastoji od atributa: ID dijete, ID klijenta te opisa dijete. Dijetu i potrebne parametre unosi doktor. Entitet Dijeta je preko atributa ID u \textit{Many-to-One} vezi s entitetom Korisnik,a preko atributa ID u \textit{One-to-One} vezi s entitetom DnevniLimit, \textit{One-to-Many} vezi s OgranicenjaNaKategorije te \textit{One-to-Many} vezi s entitetom OgranicenjaNaProizvode.
				
				\begin{longtabu} to \textwidth {|X[7, l]|X[6, l]|X[20, l]|}
					
					\hline \multicolumn{3}{|c|}{\textbf{Dijeta}}	 \\[3pt] \hline
					\endfirsthead
					
					\hline \multicolumn{3}{|c|}{\textbf{Dijeta}}	 \\[3pt] \hline
					\endhead
					
					\hline 
					\endlastfoot
					
					\colorbox{LightGreen}{ID} & INT	&  jedinstveni identifikator dijete \\ \hline
					\colorbox{LightBlue}{IDklijenta} & VARCHAR & jedinstveni identifikator klijenta\\ \hline
					Opis & VARCHAR & opis dijete\\ \hline
					
				\end{longtabu}
				
				\textbf{OgraničenjaNaProizvode} Dijeta se može definirati s ograničenjima na određene proizvode. Entitet Ograničenja se sastoji od atributa: ID dijete i ID proizvoda. Ovaj entitet je preko atributa IDdijete u \textit{Many-to-One} vezi s entitetom Dijeta te preko atributa IDproizvoda u \textit{Many-to-Many} vezi s entitetom Proizvod.
				
				\begin{longtabu} to \textwidth {|X[7, l]|X[6, l]|X[20, l]|}
					
					\hline \multicolumn{3}{|c|}{\textbf{OgraničenjaNaProizvode}}	 \\[3pt] \hline
					\endfirsthead
					
					\hline \multicolumn{3}{|c|}{\textbf{OgraničenjaNaProizvode}}	 \\[3pt] \hline
					\endhead
					
					\hline 
					\endlastfoot
					
					\colorbox{LightGreen}{IDdijete} & INT	&  jedinstveni identifikator dijete \\ \hline
					\colorbox{LightGreen}{IDproizvoda} & INT & jedinstveni identifikator proizvoda\\ \hline
					
				\end{longtabu}
				
				\textbf{OgraničenjaNaKategorije} Dijeta se može definirati s ograničenjima na određene kategorije proizvoda. Entitet Ograničenja se sastoji od atributa: ID dijete i ID kategorije. Ovaj entitet je preko atributa IDdijete u \textit{Many-to-One} vezi s entitetom Dijeta, a preko atributa IDkategorije u \textit{Many-to-Many} vezi s entitetom KategorijeProizvoda.
				
				\begin{longtabu} to \textwidth {|X[7, l]|X[6, l]|X[20, l]|}
					
					\hline \multicolumn{3}{|c|}{\textbf{OgraničenjaNaKategorije}}	 \\[3pt] \hline
					\endfirsthead
					
					\hline \multicolumn{3}{|c|}{\textbf{OgraničenjaNaKategorije}}	 \\[3pt] \hline
					\endhead
					
					\hline 
					\endlastfoot
					
					\colorbox{LightGreen}{IDdijete} & INT	&  jedinstveni identifikator dijete \\ \hline
					\colorbox{LightGreen}{IDkategorije} & INT & jedinstveni identifikator kategorije\\ \hline
					
				\end{longtabu}
				
				\textbf{DnevniLimit} Dijeta se može definirati dnevnim limitom za određene nutritivne vrijednosti proizvoda. Entitet DnevniLimit se sastoji od atributa: ID dijete, limit masnoće, limit ugljikohidrata, limit šećera, limit bjelančevina i limit soli. Ovaj entitet je preko atributa IDdijete u \textit{One-to-One} vezi s entitetom Dijeta.
				
				\begin{longtabu} to \textwidth {|X[15, l]|X[6, l]|X[20, l]|}
					
					\hline \multicolumn{3}{|c|}{\textbf{DnevniLimit}}	 \\[3pt] \hline
					\endfirsthead
					
					\hline \multicolumn{3}{|c|}{\textbf{DnevniLimit}}	 \\[3pt] \hline
					\endhead
					
					\hline 
					\endlastfoot
					\colorbox{LightGreen}{IDdijete} & INT	&  jedinstveni identifikator dijete \\ \hline
					limitEnergije & DOUBLE & limit energije u kJ 	\\ \hline 
					LimitMasnoće & DOUBLE & limit masnoća u sastavu proizvoda\\ \hline
					LimitZasMasneKiselina & DOUBLE & limit zasićenih masnih kiselina\\ \hline
					LimitUgljikohidrata & DOUBLE & limit ugljikohidrata u sastavu proizvoda\\ \hline
					LimitŠećera & DOUBLE & limit šećera u sastavu proizvoda\\ \hline
					LimitBjelančevina & DOUBLE & limit bjelančevina u sastavu proizvoda\\ \hline
					LimitSoli & DOUBLE & limit soli u sastavu proizvoda\\ \hline	
					
				\end{longtabu}
				
				\textbf{UneseneNutrVrijednosti}  Dijeta se može definirati dnevnim limitom za određene nutritivne vrijednosti proizvoda. Entitet UneseneNutrVrijednosti prati dnevni unos pojedinih nutritivnih vrijednosti. Atributi su: ID korisnika, datum, te unesena energija, masnoće, zasićene masne kiseline, ugljikohidrati, šećeri, bjelančevine i sol. Ovaj entitet je preko atributa IDklijenta u \textit{Many-to-One} vezi s entitetom Korisnik.
				
				\begin{longtabu} to \textwidth {|X[10, l]|X[6, l]|X[20, l]|}
					
					\hline \multicolumn{3}{|c|}{\textbf{UneseneNutrVrijednosti}}	 \\[3pt] \hline
					\endfirsthead
					
					\hline \multicolumn{3}{|c|}{\textbf{UneseneNutrVrijednosti}}	 \\[3pt] \hline
					\endhead
					
					\hline 
					\endlastfoot
					\colorbox{LightGreen}{IDklijenta} & VARCHAR	& jedinstveni identifikator klijenta  \\ \hline
					\colorbox{LightGreen}{Datum} & DATETIME	& datum \\ \hline
					Energija & DOUBLE & unesena energija u kJ 	\\ \hline 
					Masnoća & DOUBLE & unesena masnoća\\ \hline
					ZasMasneKiseline & DOUBLE & unesene zasićene masne kiseline\\ \hline
					Ugljikohidrati & DOUBLE & uneseni ugljikohidrati\\ \hline
					Šećeri & DOUBLE & unesena šećeri\\ \hline
					Bjelančevine & DOUBLE & unesene bjelančevine\\ \hline
					Sol & DOUBLE & unesene soli\\ \hline
					
				\end{longtabu}
				
				\textbf{KonzumiraniProizvodi} Entitet KonzumiraniProizvodi se sastoji od atributa: ID klijenta, datum te ID proizvoda. Ovaj entitet je preko atributa IDproizvoda u \textit{Many-to-Many} vezi s entitetom Proizvod te preko atributa IDklijenta u \textit{Many-to-One} vezi s entitetom Korisnik.
				
				\begin{longtabu} to \textwidth {|X[8, l]|X[6, l]|X[20, l]|}
					
					\hline \multicolumn{3}{|c|}{\textbf{KonzumiraniProizvodi}}	 \\[3pt] \hline
					\endfirsthead
					
					\hline \multicolumn{3}{|c|}{\textbf{KonzumiraniProizvodi}}	 \\[3pt] \hline
					\endhead
					
					\hline 
					\endlastfoot
					\colorbox{LightGreen}{IDklijenta} & INT	& jedinstveni identifikator klijenta\\ \hline
					\colorbox{LightGreen}{Datum} & DATETIME	& datum\\ \hline
					\colorbox{LightGreen}{IDproizvoda} & INT & jedinstveni identifikator proizvoda \\ \hline
					
				\end{longtabu}
			
			\subsection{Dijagram baze podataka}
				\begin{figure}[H]
					\includegraphics[width=\textwidth,height=\textwidth,keepaspectratio]{slike/dijagramBaze.png}
					\centering
					\caption{Dijagram baze podataka}
					\label{fig:promjene}
				\end{figure}
			
			\eject
			
			
		\section{Dijagram razreda}
		
			\textit{Potrebno je priložiti dijagram razreda s pripadajućim opisom. Zbog preglednosti je moguće dijagram razlomiti na više njih, ali moraju biti grupirani prema sličnim razinama apstrakcije i srodnim funkcionalnostima.}\\
			
			\textbf{\textit{dio 1. revizije}}\\
			
			\begin{figure}[H]
			\includegraphics[scale=0.65]{slike/Controlleri.PNG}
			\centering
			\caption{Dijagram razreda - dio Controlleri}
			\label{fig:promjene}
			\end{figure}

			\begin{figure}[H]
			\includegraphics[scale=0.6]{slike/Modeli.PNG}
			\centering
			\caption{Dijagram razreda - dio Modeli}
			\label{fig:promjene}
			\end{figure}
			
			\textbf{\textit{dio 2. revizije}}\\			
			
			\textit{Prilikom druge predaje projekta dijagram razreda i opisi moraju odgovarati stvarnom stanju implementacije}
			
			
			
			\eject
		
		\section{Dijagram stanja}
			
			git s
			\textbf{\textit{dio 2. revizije}}\\
			
			\textit{Potrebno je priložiti dijagram stanja i opisati ga. Dovoljan je jedan dijagram stanja koji prikazuje \textbf{značajan dio funkcionalnosti} sustava. Na primjer, stanja korisničkog sučelja i tijek korištenja neke ključne funkcionalnosti jesu značajan dio sustava, a registracija i prijava nisu. }
			
			
			\eject 
		
		\section{Dijagram aktivnosti}
			
			\textbf{\textit{dio 2. revizije}}\\
			
			 \textit{Potrebno je priložiti dijagram aktivnosti s pripadajućim opisom. Dijagram aktivnosti treba prikazivati značajan dio sustava.}
			
			\eject
		\section{Dijagram komponenti}
		
			\textbf{\textit{dio 2. revizije}}\\
		
			 \textit{Potrebno je priložiti dijagram komponenti s pripadajućim opisom. Dijagram komponenti treba prikazivati strukturu cijele aplikacije.}